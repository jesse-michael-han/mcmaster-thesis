\label{chap-ultracategories}
In this chapter we provide the necessary background on Makkai's theory of ultracategories.

\section{Introduction}
\definition{\label{def-ultraproduct-of-sets}An \tbf{ultraproduct} of a family $(A_i)_{i \in I}$ of non-empty sets with respect to a non-principal ultrafilter $\mc{U}$ on $I$ is the set
$$
\prod_{i \to \mc{U}} A_i \dfeq \bigslant{\prod_{i \in I} A_i}{\sim_{\mc{U}}},
$$
where $(x_i)_{i \in I} \sim_{\mc{U}} (y_i)_{i \in I}$ if and only if $\{j \in I \stbar x_i = y_i\} \in \mc{U}$. Given a representative $(x_i)_{i \in I}$ of a $\mc{U}$-class, we write $[x_i]_{i \to \mc{U}}$ for its $\sim_{\mc{U}}$-class.

Inside the category $\mbf{Set}$, this definition can be recast as the filtered colimit
$$
\prod_{i \to \mc{U}} A_i \dfeq \colim \left(\prod_{i \in P}A_i \right)_{P \in \mc{U}},
$$
where the transition maps in the filtered diagram $\prod_{i \in P} A_i \to \prod_{i \in P'} A_i$ are the projections to the coordinates $P \cap P'$. This correctly handles the possibility that on a $\mc{U}$-small set of indices, the sets $A_i$ are empty.
}

However, it is safe to assume that for the remainder of this document, we will ignore empty models, and so the definition of an ultraproduct of sets can be taken to be the first one.

\definition{\label{def-pre-ultracategory-set}When computing an ultraproduct of sets $\prod_{i \to \mc{I}} A_i$, we will follow the conventions:
\begin{enumerate}
\item Whenever we form an $I$-indexed product $\prod_{i \in I} A_i$, we will think of each sequence $(a_i)_{i \in I}$ as the set $\{(i, a_i)\}$ and we will always write $\prod_{i \in I} A_i$ as the set of those sequences: $\{\{(i, a_i)\}\}$.
  \item Whenever we have a set $X$ and an equivalence relation $E \rightrightarrows X$, we will always write the quotient $X/E$ as the set of (literal) equivalence classes of $X$.
\end{enumerate}

With these conventions in place, we know \emph{exactly} what set is the ultraproduct of a given family of sets; by applying the ultraproduct construction to the graphs of functions $(X_i \to Y_i)_{i \in I}$, we also know exactly how to take ultraproducts of functions. This all determines \tbf{ultraproduct functors} $[\mc{U}] : \Set^I \to \Set$, for every $I$ and every ultrafilter $\mc{U}$ on $I$.}

Since $\Mod(T)$ is the category of elementary functors (which we think of as pretopos morphisms) $\mbf{Pretop}(\Def(T), \Set)$, once we have specified how to take ultraproducts in $\Set$, this tells us how to define ultraproducts of models ``pointwise'':

\definition{
  \label{def-ultraproduct-of-models}
  Let $(M_i)_{i \in I}$ be an $I$-indexed sequence of models of $T$. We define the \tbf{ultraproduct of models} $\prod_{i \to \mc{U}} M_i$ to be the following elementary functor $\Def(T) \to \Set$: on objects $A \in \Def(T)$, put
  $$
\left(\prod_{i \to \mc{U}} M_i\right)(A) \dfeq \prod_{i \to \mc{U}} \left(M_i(A)\right),
$$
so that we have defined where $\prod_{i \to \mc{U}} M_i$ sends the object $A$ to be precisely the ultraproduct in $\Set$ of where each elementary functor $M_i$ sends $A$.

This determines where $\prod_{i \to \mc{U}} (M_i)$ sends maps $f : A \to B$ in $\Def(T)$, by treating $f$ as its graph relation.
  }

  As pointed out by Makkai in \cite{makkai-sdfol}, the content of the {\L}os theorem (see \ref{thm-los-theorem}) is that the previous definition of an ultraproduct of models is still an ultraproduct of models (and this boils down to showing that the ultraproduct functors on $\Set$ are elementary functors $\Set^I \to \Set$). However, in $\Mod(T)$, ultraproducts of models admit no nice definition in terms of a filtered colimit of infinite products as when we were computing ultraproducts in $\Set$---because infinite products of models might not exist.

  However, since ultraproducts of models are still computed sort-by-sort (indeed, definable set-by-definable set), one might believe that there is some residual ``niceness'' from $\mbf{Set}$ manifesting in how the ultraproducts of models interact with each other.

The purpose of the notion of ultracategory, modeled after $\Mod(T)$, is to formalize this notion of a category equipped with extra structure coming from a ``nice'' notion of taking ultraproducts of its objects. In particular, since in $\Set$, ultraproducts are a combination of products and filtered colimits, there are purely formal ``comparison maps'' between ultraproducts arising from the universal properties of products and filtered colimits, and we will see that part of the extra structure includes naming these ``comparison maps''. Functors which preserve this extra structure are called ultrafunctors, and ultrafunctors between ultracategories $X : \mbf{K} \to \mbf{K}'$ will generalize the reduct functors $\Mod(T') \to \Mod(T)$ induced by an interpretation $T \to T'$.

Makkai's duality theorem \cite{makkai-sdfol} tells us that there is a dual adjunction between first-order theories (pretoposes) and ultracategories, in fact given by homming into a dualizing object $\mbf{Set}$. Strong conceptual completeness says that the counit of this adjunction is an equivalence, i.e. that a pretopos $T$ is equivalent to the category of ultrafunctors $\mbf{Ult}(\Mod(T), \mbf{Set})$.

\section{Basic notions}
\subsection{The {\L}os theorem}
To every non-principal ultrafilter $\mc{U}$ on an indexing set $I$, we have fixed an ultraproduct functor
$$
[U] : \mbf{Set}^I \to \mbf{Set}, \hspace{5mm} (X_i)_{i \in I} \mapsto \prod_{i \to \mc{U}} X_i.
$$

The starting point is the {\L}os ultraproduct theorem, which we rephrase in terms of the ultraproduct functors in $\mbf{Set}$:

\theorem{\label{thm-los-theorem}\emph{({\L}os theorem)} Let $I$ be an indexing set and $U$ an ultrafilter on $I$. Then the ultraproduct functor $[U] : \mbf{Set}^I \to \mbf{Set}$ is elementary.}

\begin{proof}
  An elementary functor preserves initial and terminal objects, pullbacks, disjoint sums, and quotients by equivalence relations\footnote{Note that this implies that images are preserved: the image of any definable function $f : X \to Y$ is the projection to $Y$ of the graph $\Gamma(f)$ of $f$, and is therefore in definable bijection with the quotient of $\Gamma(f)$ by the definable equivalence relation $(x,y) \simeq (x', y') \iff y = y'$.}.
\begin{itemize}
\item  Initial objects: a product of the empty set is the empty set, and a quotient of the empty set is the empty set.

\item  Terminal objects: a product of terminal objects is terminal, and the quotient of a singleton is a singleton.

\item  Pullbacks: a product of pullbacks is a pullback, and finite limits commute with filtered colimits.

\item  Disjoint sums: a product of disjoint sums is a disjoint sum of products, and colimits commute with colimits.

\item  Quotients: a product of quotients $X_i/E_i$ is a quotient of products $\prod_I X_i / \prod_I E_i$, and colimits commute with colimits.
  \end{itemize}
\end{proof}

\corollary{Let $(M_i)_{i \in I}$ be an $I$-indexed family of $\mc{L}$-structures. For each $\mc{L}$-formula $\varphi(x)$, each element $\ol{a}$ of the ultraproduct $\bigslant{\prod_{i \in I} M_i}{\mc{U}},$ $$\bigslant{\prod_{i \in I} M_i}{\mc{U}} \models \varphi[\ol{a}] \iff \{i \in I \stbar M_i \models \varphi[a_i]\} \in \mc{U}.$$}

\begin{proof}
By \ref{thm-los-theorem} the ultraproduct functor is elementary, so that the process of taking points inside a model of a definable set commutes with taking ultraproducts. In symbols, $$\left(\prod_{i \in I} M_i / \mc{U}\right)(X) \simeq \prod_{i \in I} M_i(X) / \mc{U}.$$ Since this is a filtered colimit, a sequence $\ol{x}$ satisfies that its germ $[\ol{x}]$ is in $\prod_{i \in I} M_i / \mc{U}(X)$ if and only if there is some $J \in \mc{U}$ such that the restriction of $\ol{x}$ to $J$ is in $\prod_{j \in J} M_j(X)$. i.e. if $x_j \in M_j(X)$ for each $j \in J$.
  \end{proof}

We recount the proof via regular ultraproducts of the compactness theorem for first-order logic. This technique will be used in various arguments later on.
  
    \fact{\parenemph{Compactness theorem for first-order logic} \label{thm-compactness} Let $T$ be a first-order theory. $T$ has a model if and only if every finite subset $T_s \undersubseteq{fin} T$ has a model.}
  \begin{proof}
Let $I$ index the finite fragments of $T$ a first-order theory. For each $i \in I$, let $P_i$ be the collection of all $j \in I$ such that viewed as finite fragments of $T$, $j \supseteq i$. The collection $F \dfeq \{P_i\}_{i \in I}$ has the finite intersection property: $P_i \cap P_{i'} = \{j \in I \stbar j \supseteq i \te{ and } j \supseteq i'\} = \{j \in I \stbar j \supseteq i \cup i'\} = P_{i \cup i'}.$  Now take a completion $\ol{F}$ of $F$ to an ultrafilter. Let $M_i$ model each finite fragment of $T$ given by $i \in I$, and consider the ultraproduct $$M^* \dfeq \displaystyle \bigslant{\prod_{I} M_i}{\ol{F}}.$$ Then for every sentence $\phi \in T$, $\phi$ is supported on some subcollection belonging to $\ol{F}$, so is satisfied in $M^*$.
\end{proof}

We also recount and prove the following useful ultraproduct characterization of elementary classes, due to Chang and Keisler:

\definition{
  \label{def-ultraroot}
  Let $M$ be an $\mc{L}$-structure. An \emph{ultraroot} of $M$ is some structure $N$ such that $N^U \simeq M$ for some non-principal ultrafilter $\mc{U}$.
  }

\fact{
  \label{thm-ultraproduct-characterization-of-elementary-classes}
  A class $\mbf{C}$ of $\mc{L}$-structures is an elementary class if and only if it is closed under isomorphisms, ultraproducts, and ultraroots.
}

\begin{proof}
  Suppose that $\mbf{C}$ is the objects $\Mod(T)_0$ of $\Mod(T)$ for some $\mc{L}$-theory $T$. Then it closed under isomorphisms, ultraproducts (by the {\L}os theorem \ref{thm-los-theorem}), and ultraroots (since diagonal embeddings into ultrapowers are elementary).

  On the other hand, suppose that $\mbf{C}$ is a class of $\mc{L}$-structures closed under isomorphisms, ultraproducts, and ultraroots. Let $T$ be the theory $$T \dfeq \bigcap_{M \in \mbf{C}} \Th_{\mc{L}}(M).$$ It suffices to show that $\mbf{C} = \Mod(T)_0$. By definition, $\mbf{C} \subseteq \Mod(T)_0$, and the inclusion $\mbf{C} \hookrightarrow \Mod(T)_0$ reflects isomorphisms. By the Keisler-Shelah isomorphism theorem \ref{thm-keisler-shelah}, the inclusion reflects elementary equivalences. Therefore, if there is an $M \in \Mod(T)_0 \backslash \mbf{C}$, its theory must not show up in $\{\Th(N) \stbar N \in \mbf{C}\}$.

  Since $T = \bigcap \{\Th(N) \stbar N \in \mbf{C}\}$, for every finite fragment $\Sigma \undersubseteq{fin} \Th(M)$, there exists an $N_{\Sigma} \in \mbf{C}$ such that $\Th(N) \models \Sigma$. (Otherwise, there is a sentence $\psi \in \Th(M) \backslash T$ such that for all $N \in \mbf{C}$, $N \models \neg \psi$, so that $\neg \psi \in T$, a contradiction).

  There is a regular ultrafilter $\mc{U}$ such that $\displaystyle \prod_{\underset{\Sigma \undersubseteq{fin} \Th(M)}{\Sigma \to \mc{U}}} N_{\Sigma} \models \Th(M)$. Since $\mbf{C}$ was closed under ultraproducts, this contradicts our assumption that $M \not \in \mbf{C}$. Therefore, $\mbf{C} = \Mod(T)$ is an elementary class.
  
\end{proof}

Finally, we state the Keisler-Shelah isomorphism theorem (though much of the time, special models arguments suffice to replace it.)

\theorem{\label{thm-keisler-shelah}(Keisler-Shelah isomorphism theorem) Two $\mc{L}$-structures are elementarily equivalent if and only if they have isomorphic ultrapowers.}

\subsection{Frayne's lemma and Scott's lemma}
In this subsection, we state Frayne's lemma and the related Scott's lemma, which will be needed for some later results. We omit the proofs (somewhat-elaborate regular ultraproduct arguments) and refer the interested reader to \cite{bell-slomson-model-theory}.

\lemma{\label{lemma-frayne}
Let $N \equiv M$ be elementarily equivalent. Then $N$ elementarily embeds into some ultrapower $M^{\mc{U}}$ of $M$.
  }

\lemma{\label{lemma-scott}
  Let $M \overset{f}{\to} N$ be an elementary map. Then there is an ultrapower $M^{\mc{U}}$ of $M$ and an elementary embedding $N \overset{g}{\to} M^{\mc{U}}$ such that the diagram
  $$
  \begin{tikzcd}[ampersand replacement = \&]
   M^{\mc{U}}        \& \\
      M \arrow{u}{\Delta_M} \arrow[swap]{r}{f}  \& N \arrow[swap]{ul}{g}
    \end{tikzcd}
    $$
    commutes.
  }

\subsection{The Beth definability theorem}
In this subsection, we state and prove a version of the Beth definability theorem, due to Bradd Hart.

\theorem{\label{thm-beth}
Let \(L_0 \subseteq L_1\) be two languages, so that \(L_1\) has no new sorts. Let \(T_1\) be a theory in \(L_1\). Let \(F\) be the reduct functor
$$
\Mod(T_1) \to \Mod(\emptyset_{L_0})
$$
(here \(\emptyset_{L_0}\) is the empty theory in \(L_0\), whose models are just all the $L_{0}$-structures.)

Suppose that we know any one of the following:

\begin{enumerate}
\item There is a theory \(T_0\) in \(L_0\) such that \(F\) factors through \(\Mod(T_0)\) with $\Mod(T_1) \to \Mod(T_0)$ an equivalence of categories.
\item $F $ is fully faithful.
\item $F $ is injective on objects.
\item $F$ is full and faithful just on automorphism groups.
\item For all $M \in \Mod(T_1)$, every $L_0$-elementary map $f : F(M) \to F(M)^{\mc{U}}$ is (uniquely lifts to) an $L_1$-homomorphism $f = \wt{f} : M \to M^{\mc{U}}$ (between $M$ and $M^{\mc{U}}$ viewed as $L_1$-structures.)
\end{enumerate}
\ul{Then}: every \(L_1\)-formula is \(T_1\)-equivalent to an \(L_0\)-formula.}

\begin{proof}
\(1\) clearly implies \(2\).

\(2\) is equivalent to \(3\): assume not \(3\). Then there are two distinct \(L_1\)-expansions \(M\) and \(N\) of the same \(L_0\)-structure \(K\), and so the identity automorphism is not in the image of $F $ restricted to \(\Hom_{L_1}(M, N)\): \(M\) and \(N\) being different must be witnessed by a single tuple \(k\) and some symbol \(R\) from \(L_1 \backslash L_0\) such that \(\models R^M(k)\) and \(\models \neg R^M(k)\). Therefore, any automorphism of \(K\) which fixes \(k\) cannot be \(L_1\)-elementary, and this negates \(2\).

Now assume not \(2\). Since taking reducts along an inclusion of languages is always faithful, this must fail to be full, and so this must be witnessed by an \(L_0\)-elementary, \(L_1\)-not-elementary map \(g : F M \to F N\). But then the pushforward \(L_1\)-structure of \(M\) along \(g\) induces an \(L_1\)-expansion of \(F N\) distinct from \(N\), which negates \(3\).

\(2\) clearly implies \(4\).

\(4\) implies \(5\): it suffices to show fullness, so let \(f : FM \to (FM)^{\mc{U}}\). We use a special models argument: by repeatedly invoking Scott's lemma, start with $FM \to (FM)^{\mc{U}}$ and obtain a diagram of iterated ultrapowers
$$
\begin{tikzcd}[ampersand replacement = \&]
FM \arrow[swap]{d}{f}  \arrow{r} \& (FM)^{\mc{U}_0} \arrow{d} \arrow{r} \& \left((FM)^{\mc{U_0}} \right)^{\mc{U}_2}\arrow{d} \arrow{r}\& \cdots\\
(FM)^{\mc{U}} \arrow{r} \arrow{ur}  \& \left((FM)^{\mc{U}} \right)^{\mc{U}_1} \arrow{r} \arrow{ur}\&  \left( \left(F(M)^{\mc{U}} \right)^{\mc{U}_1} \right)^{\mc{U}_3} \arrow{r} \arrow{ur}\& \cdots
  \end{tikzcd}
  $$
so that the vertical arrows become an isomorphism $f_{\omega} : \mbf{M}_1 \to \mbf{M}_2$ in the limit which extends $f$. The diagonal arrows become an isomorphism $g_{\omega} \mbf{M}_2 \to \mbf{M}_1$ in the limit, and from the commutativity of the diagram at every stage, $g_{\omega} \circ f_{\omega}$ becomes an automorphism of $\mbf{M}_1$ which extends $f$. Then $g_{\omega} \circ f_{\omega}$ lifts to an $L_1$-automorphism. Restricting $g_{\omega} \circ f_{\omega}$ from $\mbf{M}_1$ to $FM$, we get that $f$ also lifts to an $L_1$-homomorphism.

Now, to show that \(5\) implies that every \(L_1\)-formula is equivalent modulo \(T\) to an \(L_0\)-formula: suppose not, so that there is an \(L_1\) formula \(\psi(x)\) such that for all \(L_0\)-formulas \(\psi(x)\), there exists an \(a\) and a \(b\) such that \(\varphi(a) \land \varphi(b)\) for all \(\varphi(x) \in L_0\), but \(\psi(a) \land \neg \psi(b)\).

This is saying that there exists a model \(M\) of \(T_1\) on which the indicator function of \(\psi(x)\) disagrees with the indicator functions of every \(\varphi(x) \in L_0\).

Now, since \(a\) and \(b\) have the same \(L_0\)-type, there exists an ultrafilter \(\mc{U}\) and a pair of maps
$$
F M \overset{f}{\underset{\Delta_{F M}}{\rightrightarrows}} (F M)^{\mc{U}}
$$
where \(f\) is some embedding which sends \(b \mapsto a\).

Since \(b\) and \(a\) have distinct \(L_1\)-types, $\wt{f}$ is not \(L_1\)-elementary. Since $\wt{f}$ is a homomorphism lifting an elementary embedding, it is an embedding, so $a$ and $b$ have the same quantifier-free $L_1$-type. Therefore, $\psi$ was not quantifier free.

This implies that every quantifier-free $L_1$ formula $\psi$ is $T$-provably equivalent to an $L_0$-formula.

Since interpretations commute with quantification, we conclude that every $L_1$-formula is $T$-provably equivalent to an $L_0$-formula.
  \end{proof}

With a little more work, we can remove the stipulation that no new sorts are added.
  
  \theorem{
    Let $L_0 \subseteq L_1$ be an inclusion of languages, possibly with new sorts. Let $T$ be an $L_1$-theory.

    Suppose that whenever $M \models T$ and $\mc{U}$ is an ultrafilter, then every elementary map $$M \restr_{L_0} \to \left(M \restr_{L_0}\right)^{\mc{U}}$$ has a unique lift to a homomorphism on $M$.

    Then every definable set $X$ of $T$ is $T$-provably equivalent (i.e. equivalent modulo $T$) to an $L_0$-imaginary sort.
  }
  \begin{proof}
    The proof proceeds via the following steps:
    \begin{enumerate}
    \item Show that under our assumptions, in every model $M \models T$, the points in $M$ of every $L_1$-sort is contained in the definable closure of the points in $M$ of the $L_0$-sorts.
    \item By a compactness argument (see proof of \ref{thm-conceptual-completeness}), every $L_1$-sort is the surjective image of an $L_0$-imaginary sort.
    \end{enumerate}

    We will spell out the first item, and trust that the reader will be able to imitate the identical compactness argument from \ref{thm-conceptual-completeness}. Suppose towards the contrapositive that there exists a model $M \models T$ such that there is an $L_1$-sort $S$ such that $M(S)$ is not in the definable closure of $M(L_0)$. Then (possibly enlarging $M$) there are two points $x,y \in M(S)$ with the same type over $M(L_0)$. So, for some ultrapower $M^{\mc{U}}$ of $M$, $\Delta_M(x)$ and $\Delta_M(y)$ are conjugate by an automorphism $\sigma$ fixing $M(L_0)$.

      Then both $\Delta_M : M \to M^{\mc{U}}$ and $\sigma \circ \Delta_M : M \to M^{\mc{U}}$ lift $\Delta : M \restr_{L_0} \to \left(M \restr_{L_0}\right)^{\mc{U}}$, which violates the assumptions of the theorem.
    \end{proof}
  
\subsection{Conceptual completeness}
In this subsection, we state the conceptual completeness theorem (7.1.8, \cite{makkai-reyes}) from Makkai-Reyes and give a proof, following (4.4, \cite{kamensky-categorical-internality}).

\theorem{
  \label{thm-conceptual-completeness}
Let $T_1$ and $T_2$ eliminate imaginaries. Let $J : \Def(T_1) \to \Def(T_2)$ be an interpretation of $T_1$ in $T_2$. Let $J^* \dfeq (- \circ J)$ be the induced functor $\Mod(T_2) \to \Mod(T_1)$.

Then $J$ is an equivalence of categories if and only if $J^*$ is.
}

\begin{proof}
  That $J^*$ is an equivalence of categories if $J $ is is purely formal, c.f. the lemma \ref{lemma-equivalent-categories-have-equivalent-functor-categories}.

  Towards the other direction, suppose $J^*$ is an equivalence of categories. We need to show that $J$ is full, faithful, and essentially surjective.

  To see that $J$ is faithful: if $f_1 \neq f_2$ for $Y_1 \overset{f_1}{\underset{f_2}{\rightrightarrows}} Y_2$ in $\Def(T_1)$, then their equalizer is not all of $Y_1$, which is to say that
  $$
\models \exists y \in Y_1 \suchthat y \not \in \opn{eq}(f_1, f_2).
$$
Since $J$ is an interpretation, as a functor it preserves finite limits, complementation, and existential quantification. Applying $J$ to the above sentence, conclude that $J(f_1) \neq J(f_2)$.



\begin{description}
\item[Claim.] If $J$ is essentially surjective, it is full.
  \begin{proof}[Proof of claim.]
If $g : J(Y_1) \to J(Y_2)$ is a definable function, then its graph $\Gamma(g)$ is a definable set $\Gamma(g) \hookrightarrow J(Y_1 \times Y_2)$. If $J$ is essentially surjective, then there is a corresponding $\Gamma(\ol{g}) \hookrightarrow (Y_1 \times Y_2)$ such that $J(\ol{g}) = g$.
    \end{proof}
  \end{description}

  So, it suffices to see that $J$ is essentially surjective.

  First, we show that to prove this, it suffices to be able to place every object of $\Def(T_2)$ inside an object coming from $T_1$:

  \begin{description}
\item[Claim.] Let $X \in \Def(T_2)$. If there exists $Y \in \Def(T_1)$ with $X \hookrightarrow J(Y)$, then there exists $\ol{X} \in \Def(T_1)$ with $J(\ol{X}) = X$.
  \begin{proof}[Proof of claim.]
Let $M$ and $N$ be two models of $T_2$. If $J^*M = J^* N$, then $M(X) = N(X)$ since $M(X) \hookrightarrow M(Y) = N(Y) \hookleftarrow N(X)$ and since $J^*$ is an equivalence (consider a lift of the identity and the corresponding naturality square for the inclusion $X \hookrightarrow J(Y)$), $M(X) = N(X)$.
\end{proof}
\end{description}

Next, we claim that if $M_2$ is any model of $T_2$, then any element $a$ of $M_2$ is definable over $J^*M_2$. Indeed, we can replace $M_2$ with a larger model such that there are two elements $a$ and $b$ which are not definable over $J^* M_2$ but which have the same type over $J^*M_2$. Then there is an ultrapower $^*M^2$ of $M^2$ and an automorphism of this ultrapower which moves $\Delta(a)$ to $\Delta(b)$. This would yield two different embeddings of $M_2$ in $^*M_2$; these agree on $J^* M_2$, which would contradict that $J^*$ was an equivalence.

We will now use a compactness argument to show that, in $\Def(T_2)$, any definable set $Y$ of $T_2$ is the image of a definable map from a definable set $J(X)$ coming from $T_1$.

So, suppose that $Y$ is not covered by any finite collection of functions whose domains lie in sorts coming from $T_1$. This means that for any finite collection of such functions, every model realizes a witness $d \in Y$ which lies outside the images of the functions. That is, after introducing a generic constant symbol $d$, the theory
$$
T' \dfeq T_2 \cup \{\neg \exists r \phi(r, d) \stbar \phi \te{ is a function whose domain lies in a $T_1$-sort}\} \cup \{d \in Y\}
$$
is finitely consistent, therefore consistent. So $T'$ has a model. But in any model $M$ of $T'$, the realization of $d$ will not be $T_2$-provably definable over $J^* M$ (since otherwise the formula $\varphi(x,y)$ which witnesses this can be restricted to a definable function whose domain is in a sort coming from $T_1$), contradicting the previous claim. Therefore, there exists some definable set $J(X)$ such that there is a definable surjection $J(X) \twoheadrightarrow Y$.

By the second claim of this proof, the kernel relation of the definable surjection $J(X) \twoheadrightarrow Y$ is in the image of $J$. Therefore, $Y$ is isomorphic to an imaginary sort of $T_1$, and since $T_1$ eliminates imaginaries, $Y$ is in the essential image of $J$.
  \end{proof}

\section{Ultracategories and ultrafunctors}


\subsection{Pre-ultracategories and pre-ultrafunctors}
\definition{\label{def-pre-ultracategory}(\cite{makkai-sdfol}, Section 1)
A \tbf{pre-ultracategory} $\ul{\mbf{S}}$ is a category $\mbf{S}$ along with specified ultraproduct functors $[U] : \mbf{S}^I \to \mbf{S}$ for every set $I$ and every non-principal ultrafilter $\mc{U}$ on $I$.
  }

  (Of course, this is not enough structure to nail down what it means to have a nice notion of being able to form ultraproducts of families of objects; there are no restrictions on what these ultraproduct functors might be, or how they interact. For example, given an pre-ultracategory, we could replace $[\mc{U}]$ for each $I$ with $[\mc{V}]$ for $\mc{V}$ some fixed principal ultrafilter, and this would still be a pre-ultracategory.)

  The prototypical pre-ultracategory is $\Set$; we have already described its ultraproduct functors.

There is an obvious notion of a structure-preserving map between pre-ultracategories.
    
\definition{\label{def-pre-ultrafunctor}(\cite{makkai-sdfol}, Section 1)
A \emph{pre-ultrafunctor} $\ul{\mbf{S}} \to \ul{\mbf{S}}'$ is a functor $X : \mbf{S} \to \mbf{S}'$ along with a specified \emph{transition isomorphism} $$\Phi_{X, U} : X \circ [U] \overset{\simeq}{\to} [U] \circ X^I,$$ for each $I$ and each $U$ an ultrafilter on $I.$ That is, we require all diagrams
$$
\begin{tikzcd}[ampersand replacement = \&]
\mbf{S}^I \arrow{rr}{X^I} \arrow[swap]{dd}{[U]_{\mbf{S}}} \& \& (\mbf{S}')^I \arrow{dd}{[U]_{\mbf{S}'}} \\
\& \& \\
\mbf{S} \arrow[swap]{rr}{X} \& \& \mbf{S}'
\end{tikzcd}
$$
to commute, where $U$ ranges over all non-principal ultrafilters on $I$ ranging over all small indexing sets. (``Ultraproducts are preserved up to the transition isomorphism $\Phi_{X,U}.$'')
}

\begin{rem} \label{rem-obvious-remark}Every functor of points $\operatorname{ev}_{\varphi(x)}$ can be canonically viewed as a pre-ultrafunctor with the transition isomorphisms $\Phi$ just the identity maps (corresponding to the equality signs in the above diagrams). \end{rem}

\remark{
  Because we only require our pre-ultrafunctors to commute with ultraproducts up to transition isomorphisms, one can have functors $X : \Mod(T) \to \mbf{Set}$ induced by taking certain (clearly non-definable) subsets of models which are isomorphic anyway by \emph{some} natural transformation of functors to a definable functor. We give an example below, which is the basis of the constructions in \ref{first-batch}.
}

\example{\label{example-simple-counterexample}
  Let $T$ be the theory of equality on an infinite set expanded by denumerably many distinct constant symbols $\{c_i\}_{i \in \omega}$. Then the functor $X : \Mod(T) \to \mbf{Set}$ which is induced by sending $$M \hspace{3mm} \mapsto \hspace{3mm} \{c_i \stbar i \te{ even}\} \cup \left(M \backslash \{c_i\}_{i \i \omega} \right)$$
  is isomorphic to the functor $\ev_{=}$, which just takes the $1$-sort of any model.
The isomorphism $X \simeq \ev_{=}$ is given by on each model $M$ by making it the identity on the omittable type $\left(M \backslash \{c_i\}_{i \i \omega} \right)$; on constants, we use any bijection $\mathbb{N} \to 2 \mbb{N}$, say $k \mapsto 2 \cdot k$.
}

We want to work with a category of pre-ultrafunctors, so we must describe what it means to have a morphism of pre-ultrafunctors.

\definition{
  \label{def-ultratransformation}(\cite{makkai-sdfol}, Section 1)
Given two pre-ultrafunctors $(X, \Phi)$ and $(X', \Phi')$, we define a map between them, called an \textbf{ultratransformation}, to be a natural transformation $\eta : X \to X'$ which satisfies the following additional property: all diagrams
$$\begin{tikzcd}[ampersand replacement = \&]
X\left(\prod_{i \to \mathcal{U}} M_i \right) \arrow{r}{\Phi_{(M_i)}} \arrow[swap]{d}{\eta_{\prod_{i \to \mathcal{U}} M_i}}   \& \prod_{i \to \mathcal{U}} X(M_i) \arrow{d}{\prod_{i \to \mathcal{U}} \eta_{M_i}}  \\
X'\left(\prod_{i \to \mathcal{U}} M_i\right) \arrow[swap]{r}{\Phi'_{(M_i)}}  \& \prod_{i \to \mathcal{U}} X'(M_i)\end{tikzcd}$$
must commute.
  }


\begin{defn} \label{def-category-of-pre-ultrafunctors} The \textbf{category of pre-ultrafunctors} $\mathbf{PUlt}(\operatorname{\mathbf{Mod}}(T), \mathbf{Set})$ comprises the following data:
  $$\mathbf{PUlt}(\operatorname{\mathbf{Mod}}(T),\mathbf{Set}) \overset{\operatorname{df}}{=} \begin{cases} \text{Objects: pre-ultrafunctors $(X,\Phi) : \operatorname{\mathbf{Mod}}(T) \to \mathbf{Set}$ } \\ \text{Morphisms: ultratransformations $\eta : (X, \Phi) \to (X, \Phi')$.}
  \end{cases}$$
 \end{defn}
  
\subsection{Ultramorphisms}
\subsubsection{Introduction}
In $\mbf{Set}$ ultraproducts are computed as certain filtered colimits, and so there are canonical comparison maps between them induced by maps between their underlying diagrams (via their universal properties).

For example, consider the terminal map $I \to 1$. This induces a diagonal map $A \to \prod_{i \in I} A$ by $a \mapsto (a,a,a, \dots)$, and this induced map extends along ultraproducts to give the diagonal map
$$
\Delta : M \hookrightarrow M^{\mc{U}}
$$
of a model into its ultrapower.

In $\Mod(T)$, these filtered colimits don't usually exist because products of models of a first-order theory don't usually exist. For example, $\Mod(T)$ doesn't see the diagonal map $A \to \prod_{i \in I} A$, only the diagonal map $\Delta : M \hookrightarrow M^{\mc{U}}$ it induces on models. So the pure category $\Mod(T)$ does not distinguish (say) $\Delta : M \hookrightarrow M^{\mc{U}}$ from any other embedding $M \hookrightarrow M^{\mc{U}}$, because there is no canonical way to obtain $\Delta$. But once we force $\Mod(T)$ to remember that ultraproducts of models are computed as certain filtered colimits in $\mbf{Set}$, then $\Delta : M \hookrightarrow M^{\mc{U}}$ is distinguished by the ultracategory $\Mod(T)$ because it arises in a canonical way.

The purpose of the notion of ultramorphisms is to name all the maps between ultraproducts in an pre-ultracategory which ``should'' arise in a canonical way. It turns out that this is enough to correct for the laxness in the definition of a pre-ultracategory: an ultracategory will be precisely a pre-ultracategory with as many ultramorphisms as possible, and after that we get Makkai's duality and strong conceptual completeness.

\subsubsection{The definition of an ultramorphism}
\definition{ (\cite{makkai-sdfol}, Section 3)
  \label{def-ultragraph}
  An \tbf{ultragraph} $\Gamma$ comprises:
  \bfenumerate{
  \item Two disjoint sets $\Gamma^f$ and $\Gamma^b$, called the sets of \tbf{free nodes} and \tbf{bound nodes}, respectively.
    
  \item For any pair $\gamma, \gamma' \in \Gamma$, there exists a set $E(\gamma, \gamma')$ of \tbf{edges}. This gives the data of a directed graph.
    
  \item For any bound node $\beta \in \Gamma^b$, we assign a triple (``ultraproduct data'') $\langle I, \mc{U}, g \rangle \dfeq \langle I_{\beta}, \mc{U}_{\beta}, g_{\beta} \rangle$ where $\mc{U}$ is an ultrafilter on $I$ and $g$ is a function $g : I \to \Gamma^f$.
    }
  }

  \definition{\label{def-ultradiagram} (\cite{makkai-sdfol}, Section 3)
    An \tbf{ultradiagram of type $\Gamma$} in a pre-ultracategory $\ul{\mbf{S}}$ is a diagram $A : \Gamma \to \mbf{S}$ assigning an object $A$ to each node $\gamma \in X$, and assigning a morphism in $\mbf{S}$ to each edge $e \in E(\gamma, \gamma')$, such that
    $$
\label{eqn-ultradiagram-condition} A(\beta) = \prod_{i \in I_{\beta}} A(g_{\beta}(i))/\mc{U}_{\beta}
    $$
    for all bound nodes $\beta \in \Gamma^b$.
 }

  Given this notion of a diagram with extra structure, there is an obvious notion of natural transformations between such diagrams which preserve the extra given structure.
  
  \definition{
    \label{def-ultradiagram-morphism}
    (\cite{makkai-sdfol}, Section 3)
    Let $A, B : \Gamma \to \mbf{S}$.
    A \tbf{morphism} of ultradiagrams $\Phi : A \to B$ is a natural transformation $\Phi$ satisfying
    $$
\label{eqn-ultradiagram-morphism-condition} \Phi_{\beta} = \prod_{i \to \mc{U}_{\beta}} \Phi_{g_{\beta}(i)}
$$
for all bound nodes $\beta \in \Gamma^b$.
}

Finally, we can define ultramorphisms.

\definition{
  \label{def-ultramorphism}
  (\cite{makkai-sdfol}, Section 3)
  Let $\Hom(\Gamma, \ul{\mbf{S}})$ be the category of all ultradiagrams of type $\Gamma$ inside $\ul{\mbf{S}}$ with morphisms the ultradiagram morphisms \ref{def-ultradiagram-morphism} defined above. Any two nodes $k, \ell \in \Gamma$ define \emph{evaluation functors} $(k), (\ell) : \Hom(\Gamma, \ul{\mbf{S}}) \rightrightarrows \mbf{S},$ by
  $$
(k) \left(A \overset{\Phi}{\to} B \right) = A(k) \overset{\Phi_k}{\to} B(k)
$$
(resp. $\ell$).

An \tbf{ultramorphism} of type $\langle \Gamma, k, \ell \rangle$ in $\ul{\mbf{S}}$ is a natural transformation $\delta : (k) \to (\ell)$.\footnote{Note that in our terminology, an ultramorphism, singular, refers to a collection of possibly many maps (the components of the natural transformation $(k) \to (\ell)$).}
}
  \subsubsection{Examples of ultramorphisms}
Let us unravel this definition for the prototypical example $\Delta : M \hookrightarrow M^{\mc{U}}$ of an ultramorphism.

\example{\label{example-ultramorphism-diagonal-map}
  Given an ultrafilter $\mc{U}$ on $I$, put:
  \begin{itemize}
  \item $\Gamma^f = \{k\}$,
  \item $\Gamma^b = \{\ell\}$,
  \item $E(\gamma, \gamma') = \emptyset$ for all $\gamma, \gamma' \in \Gamma$,
  \item $\langle I_{\ell}, \mc{U}_{\ell}, g_{\ell} \rangle = \langle I, \mc{U}, g \rangle$ where $g$ is the constant map to $k$ from $I$.
  \end{itemize}

  By the ultradiagram condition \ref{eqn-ultradiagram-condition}, an ultradiagram $A$ of type $\Gamma$ in $\ul{\mbf{S}}$ is determined by $A(k)$, with $A(\ell) = A(k)^{\mc{U}}$.

  By the ultradiagram morphism condition \ref{eqn-ultradiagram-morphism-condition}, an ultramorphism of type $\langle \Gamma, k, \ell \rangle$ must be a collection of maps $\left(\delta_M : M \to M^{\mc{U}} \right)_{M \in \Mod(T)}$ which make all squares of the form
  $$
  \begin{tikzcd}[ampersand replacement = \&]
  M^{\mc{U}} \arrow{r}{f^{\mc{U}}}  \& N^{\mc{U}} \\
  M \arrow{u}{\Delta_M} \arrow[swap]{r}{f}  \& N \arrow[swap]{u}{\Delta_N}
  \end{tikzcd}
  $$
  commute. It is easy to check that setting $\delta_M = \Delta_M$ the diagonal embedding gives an ultramorphism.}

  \definition{
    \label{def-generalized-diagonal-embeddings} \label{def-delta-g}
    The next least complicated example of an ultramorphism are the \tbf{generalized diagonal embeddings}. Here is how they arise: let $g : I \to J$ be a function between two indexing sets $I$ and $J$. $g$ induces a pushforward map $g_* : \beta I \to \beta J$ between the spaces of ultrafilters on $I$ and $J$, by $g_* \mc{U} \dfeq \{P \subseteq J \stbar g^{-1}(P) \in \mc{U}$. Fix $\mc{U} \in \beta I$ and put $\mc{V} \dfeq g_* \mc{U}$. Let $(M_j)_{j \in J}$ be a $J$-indexed family of models.

    Then there is a canonical ``fiberwise diagonal embedding''
    $$
\Delta_g : \prod_{j \to \mc{V}} M_j \to \prod_{i \to \mc{U}} M_{g(i)}
$$
given on $[a_j]_{j \to \mc{V}}$ by replacing each entry $a_j$ with $g^{-1}(\{a_j\})$-many copies of itself.

In terms of the definition \ref{def-ultramorphism} of an ultramorphism, the free nodes are $J$, and there are two bound nodes $k$ and $\ell$. To $k$ we assign the triple $\langle J, \mc{V}, \id_J \rangle$ and to $\ell$ we assign the triple $\langle I, \mc{U}, g \rangle$. Then $\Delta_g$ induces an ultramorphism $(k) \to (\ell)$.
}


\subsection{What it means for a pre-ultrafunctor to preserve an ultramorphism}
Given the protytpical diagonal embedding ultramorphisms $\Delta_M$, we can say what it means that a pre-ultrafunctor $(X, \Phi)$ \emph{preserves} diagonal embeddings.

\definition{
  \label{def-delta-functor}
  We say that a pre-ultrafunctor \ref{def-pre-ultrafunctor} $(X,\Phi)$ is a \tbf{$\Delta$-functor} if for every $I$, for every $\mc{U}$, and for every $M$ and the diagonal embedding $M \overset{\Delta_M}{\longrightarrow} M^{\mc{U}}$, the diagram
  $$
  \begin{tikzcd}[ampersand replacement = \&]
    \& X\left(M^{\mc{U}}\right) \arrow{dd}{\Phi_{(M)}}\\
X(M) \arrow{ur}{X(\Delta_M)} \arrow[swap]{dr}{\Delta_{X(M)}}    \& \\
    \& X(M)^{\mc{U}}
    \end{tikzcd}
    $$
    commutes.
  }

  Analogously, we can define what it means for $(X,\Phi)$ to preserve a general ultramorphism \ref{def-ultramorphism}. Let $(X,\Phi) : \ul{\mbf{K}} \to \ul{\mbf{S}}$ be a pre-ultrafunctor between the pre-ultracategories $\ul{\mbf{K}}$ and $\ul{\mbf{S}}$, and let $\delta$ be an ultramorphism in $\mbf{K}$ and $\delta'$ an ultramorphism in $\mbf{S}$, both of type $\langle \Gamma, k, \ell \rangle$.

Recall that in the terminology of the definition \ref{def-ultramorphism}, $\delta$ is a natural transformation $(k) \overset{\delta}{\to} (\ell)$ of the evaluation functors $$(k), (\ell) : \Hom(\Gamma, \mbf{K}) \to \mbf{K}.$$ (Resp. $\delta'$, $\mbf{S}$.)

One would like to be able to say that for any ultradiagram $\mscr{M} \in \Hom(\Gamma, \mbf{K})$, we can apply $X$ to produce a ``pushforward'' ultradiagram $X \circ \mscr{M}$ in $\Hom(\Gamma, \mbf{S})$. However, since we defined ultradiagrams ``strictly'' (by requiring that there is a literal equality between $\mscr{M}(\beta)$ for $\beta$ a bound node with ultraproduct data $(I, \mc{U}, g)$ and $\prod_{i \to \mc{U}} \mscr{M}(g(i))$), this only happens if $(X, \Phi)$ is a strict preultrafunctor (c.f. \ref{lemma-easy-calculation}). So, we will do the next best thing and ``strictify'' $X \circ \mscr{M}$.

\definition{
  \label{def-XM} (\cite{makkai-sdfol}, Section 3)
  Let $(X,\Phi) : \ul{\mbf{K}} \to \ul{\mbf{S}}$ be a pre-ultrafunctor between the pre-ultracategories $\ul{\mbf{K}}$ and $\ul{\mbf{S}}$. We define an ultradiagram $X \mscr{M} : \Gamma \to \ul{\mbf{S}}$ as follows:
  \begin{enumerate}
  \item If $\gamma$ is a free node, then $X \mscr{M}(\gamma) \dfeq X \circ \mscr{M}(\gamma)$.
    \item If $\beta$ is a bound node with ultraproduct data $(I, \mc{U}, g),$ then $X \mscr{M}(\beta) \dfeq \prod_{i \to \mc{U}} X \circ \mscr{M}(g(i))$.
    \end{enumerate}

    There is an obvious natural isomorphism of functors $\nu : X \circ \mscr{M} \simeq X \mscr{M}$ whose component $\nu_{\beta}$ at a bound node $\beta$ with ultraproduct data $(I, \mc{U}, g)$ is the appropriate component of the transition isomorphism $\nu_{\beta} \dfeq \Phi_{X \circ \mscr{M}(g(i))}$ and whose component at a free node $\gamma$ is just the identity map $\nu_{\gamma} \dfeq \id_{X \circ \mscr{M}(\gamma)}$.
  }

\definition{\label{def-preservation-of-ultramorphism}(\cite{makkai-sdfol}, Section 3)
Let $(X,\Phi) : \ul{\mbf{K}} \to \ul{\mbf{S}}$ be a pre-ultrafunctor between the pre-ultracategories $\ul{\mbf{K}}$ and $\ul{\mbf{S}}$, and let $\delta$ be an ultramorphism in $\mbf{K}$ and $\delta'$ an ultramorphism in $\mbf{S}$, both of type $\langle \Gamma, k, \ell \rangle$.

We say that $X$ \tbf{carries $\delta$ into $\delta'$} (prototypically, $\delta$ and $\delta'$ will both be canonically defined in the same way in both $\mbf{K}$ and $\mbf{S}$ and in this case we say that $\delta$ has been preserved) if for every ultradiagram $\mscr{M} \in \Hom(\Gamma, \mbf{K})$, the diagram
$$
\begin{tikzcd}[ampersand replacement = \&]
X \left(\mscr{M}(k)\right) \arrow{r}{X(\delta_{\mscr{M}})} \arrow[swap]{d}{\nu_{k}} \& X\left(\mscr{M}(\ell)\right) \arrow{d}{\nu_{\ell}} \\
(X \mscr{M})(k) \arrow[swap]{r}{\delta'_{X\mscr{M}}} \& (X \mscr{M})(\ell)
\end{tikzcd}
$$
commutes.
}

\subsection{The definitions of ultracategory and ultrafunctor}
Denote the class of all ultramorphisms in $\mbf{Set}$ by $\Delta(\mbf{Set})$.

\definition{
  \label{def-ultracategory} (\cite{makkai-sdfol}, Section 3)
  An \tbf{ultracategory} $\mbf{K}$ is a pre-ultracategory (c.f. \ref{def-pre-ultracategory}) $K$ whose ultramorphisms are ``fibered over'' those of $\mbf{Set}$: we additionally require a specification of an ultramorphism $\delta_{\mbf{K}}$ associated with any $\delta \in \Delta(\mbf{Set})$ such that $\delta_{\mbf{K}}$ is of the same type $(\Gamma, k, \ell)$ as $\delta$.}

  \definition{
    \label{def-ultrafunctor} (\cite{makkai-sdfol}, Section 3)
    We define an \tbf{ultrafunctor} $X : \mbf{K} \to \mbf{S}$ between ultracategories $\mbf{K}$, $\mbf{S}$ to be a pre-ultrafunctor (c.f. \ref{def-pre-ultrafunctor}) which respects the fibering over $\mbf{Set}$: for every $\delta \in \Delta(\mbf{Set})$, $X$ carries $\delta_{\mbf{K}}$ into $\delta_{\mbf{S}}$ (in the sense of the definition \ref{def-preservation-of-ultramorphism} above) for all $\delta \in \Delta(\mbf{Set})$.
  }

  \begin{defn} \label{def-ultrafunctor-morphism} A map between ultrafunctors is just an ultratransformation \ref{def-pre-ultrafunctor} of the underlying pre-ultrafunctors.

    We write $\mathbf{Ult}(\operatorname{\mathbf{Mod}}(T), \mathbf{Set})$ for the category of ultrafunctors $\operatorname{\mathbf{Mod}}(T) \to \mathbf{Set}$.
    \end{defn}

\section{The ultracategory structure on $\Mod(T)$}
$\Mod(T)$ is canonically equipped with the structure of a pre-ultracategory by ``lifting'' the canonical pre-ultracategory structure on $\Set$: an ultraproduct of models is just the ultraproduct of the underlying sets of the models.

We now describe how to additionally canonically equip $\Mod(T)$ with the structure of an ultracategory (``lifting'', as before, the canonical ultracategory structure on $\Set$).

\lemma{\label{lemma-easy-calculation}
  Let $\mscr{M} : \Gamma \to \Mod(T)$ be an ultradiagram. Let $X : \Mod(T) \to \Set$ be a strict pre-ultrafunctor. Then:
  \begin{enumerate}
  \item $X \circ \mscr{M} : \Gamma \to \Set$ is an ultradiagram.
    \item if $\eta : \mscr{M}_1 \to \mscr{M}_2$ is an ultradiagram morphism, then $X\eta \dfeq \{X\left(\eta_{\gamma}\right)\}_{\gamma \in \Gamma}$ is an ultradiagram morphism $X \circ \mscr{M}_1 \to X \circ \mscr{M}_2$.
    \end{enumerate}
    }

  \begin{proof}
        If $\beta$ is a bound node of $\Gamma$ with ultraproduct data $(I, \mc{U}, g)$, then
\begin{enumerate}\item
    $$
X \circ \mscr{M}(\beta) = X\left( \prod_{i \to \mc{U}} \mscr{M}(g(i))\right) = \prod_{i \to \mc{U}} X \circ \mscr{M}(g(i)),
$$
whence strictness of the pre-ultrafunctor $X$. Thus the ultradiagram condition \ref{eqn-ultradiagram-condition} is satisfied.
\item $$X \left(\eta_{\beta}\right) = X \left( \prod_{i \to \mc{U}} \eta_{g(i)} \right) = \prod_{i \to \mc{U}} X\left(\eta_{g(i)}\right),$$ whence strictness of the pre-ultrafunctor $X$. Thus the ultradiagram morphism condition \ref{eqn-ultradiagram-morphism-condition} is satisfied.
  \end{enumerate}
    \end{proof}
  
\definition{
  \label{def-mod-t-ultracategory}(\cite{makkai-sdfol}, Section 3)
  We make the pre-ultracategory $\Mod(T)$ into an ultracategory by specifying, for each ultramorphism $\delta$ in $\Delta(\Set)$ of type $(\Gamma, k, \ell)$, for every ultradiagram $\mscr{M} : \Gamma \to \Mod(T)$, and for every object $A \in \Def(T)$,
  $$
\left(\left(\delta_{\Mod(T)}\right)_{\mscr{M}}\right)_A \dfeq \delta_{\ev_A \circ \mscr{M}}.
$$
Remembering that $\delta_{\Mod(T)}$ is supposed to be a natural transformation of evaluation functors on ultradiagrams, and elementary embeddings are natural transformations, the equation displayed above reads: the component at the definable set $A$ of the component at $\mscr{M}$ of the ultramorphism $\delta_{\mscr{Mod}(T)}$ is defined to be the component at the ultradiagram $\ev_A \circ \mscr{M}$ of $\delta$.

It is easy to verify that $\delta_{\mbf{Mod}(T)}$ so defined is an ultramorphism, using the previous lemma.
}

\proposition{
  Let $A \in \Def(T)$. Then the strict pre-ultrafunctor $\ev_A : \Mod(T) \to \Set$ is an ultrafunctor.
}

\begin{proof}
  Setting up the preservation of ultramorphisms condition \ref{def-preservation-of-ultramorphism}, it remains to check that the diagram
  $$
  \begin{tikzcd}[ampersand replacement = \&]
    \ev_A(\mscr{M}(k)) \arrow[equals]{d} \arrow{r}{\ev_A\left(\left(\delta_{\Mod(T)}\right)_{\mscr{M}} \right)} \& \ev_A(\mscr{M}(\ell)) \arrow[equals]{d}\\
\ev_A\left(\mscr{M}(k)\right)  \arrow[swap]{r}{\delta_{\ev_A \circ \mscr{M}}}\& \ev_A(\mscr{M}(\ell))
  \end{tikzcd}
  $$ commutes. So, $$\ev_A \left(  \left(\delta_{\Mod(T)}\right)_{\mscr{M}}\right) = \left(\left(\delta_{\Mod(T)}\right)_{\mscr{M}} \right)_{A},$$
  which was defined above to be $\delta_{\ev A \circ \mscr{M}}$.
\end{proof}

\section{Strong conceptual completeness}
 There is a canonical evaluation functor $$ \wt{\ev} : \Def(T) \to \Ult(\Mod(T), \Set) $$ sending each definable set $A \in T$ to its corresponding ultrafunctor $\wt{\ev}_A$, and we now have the following picture of factorizations of the original evaluation map $\ev : \Def(T) \to [\Mod(T) \to \Set]$:
  $$
  \begin{tikzcd}[ampersand replacement = \&]
  \Def(T) \arrow[swap, bend right]{ddr}{\ev} \arrow[swap]{dr}{\wh{\ev}} \arrow{r}{\wt{\ev}} \& \Ult(\Mod(T),\Set) \arrow[hook]{d} \\
    \& \PUlt(\Mod(T),\Set) \arrow{d} \\
    \& \left[\Mod(T), \Set\right]
    \end{tikzcd}
  $$
  Now, we can state strong conceptual completeness.
  
  \theorem{\label{thm-scc}(\cite{makkai-sdfol}, Section 4) $\wt{\ev} : \Def(T) \to \Ult(\Mod(T), \Set)$ is an equivalence of categories.}
  
% Write $e$ for $\wt{\ev}$ and $T'$ for the pretopos $\Ult(\Mod(T), \Set)$.

% \definition{\label{def-conservative}
%   Let $e : T \to T'$ be a pretopos morphism. For any $A \in T$, $e$ induces a homomorphism of subobject lattices
%   $$ e^{(A)} : \Sub_T(A) \to \Sub_{T'}(e(A)) $$.
%   We say that:
%   \begin{enumerate}
%   \item $e$ is \tbf{conservative} if for all $A$, $e^{(A)}$ is injective.
%   \item $e$ is \tbf{full on subobjects} if for all $A$, $e^{(A)}$ is surjective.
%     \item $e$ \tbf{finitely covers} $T'$ if for all $B \in T'$, there exist finitely many $A_1, \dots, A_n$, subobjects $B_i \hookrightarrow e(A_i)$ for each $i \leq n$, and maps $\phi_i : B_i \to B$ such that the images of the $\phi_i$ cover $B$.
%     \end{enumerate}
%   }

%   \definition{\label{def-reflects-isomorphisms}
% A functor $F : \mbf{C} \to \mbf{D}$ \tbf{reflects isomorphisms} if for every $f \in C_1$, $F(f)$ being an isomorphism implies that $f$ was an isomorphism.
%     }
  
% \remark{In \cite{johnstone-elephant}, a functor is said to be conservative if it reflects isomorphisms. It is easy to show these two notions coincide for pretoposes (not necessarily Boolean).}
  
%   \lemma{
% Let $T \overset{e}{\to} T'$ be a pretopos morphism. $e$ is conservative (in the sense of \ref{def-conservative} if and only if $e$ reflects isomorphisms.
% }

% \begin{proof}
%   Suppose first that $e$ is conservative. Let $A \overset{f}{\to} B$ be a map in $T$. Suppose that $e(f)$ is an isomorphism. If $f$ is not an isomorphism, then $\im(f) \neq B$. Since $e^{(B)}$ is injective and pretopos morphisms commute with images, this means $\im(e(f)) \neq e(B)$. Therefore, $e(f)$ is not an isomorphism either, a contradiction. 

% On the other hand, if $e$ is not conservative, then for some object $A \in T$, there are subobjects $A_1 \neq A_2$ of $A$ such that $e(A_1) = e(A_2)$. Since $A_1 \neq A_2$, the inclusion of subobjects $A_1 \hookrightarrow A_1 \land A_2$ (say) is not an isomorphism. Since $e(A_1) = e(A_2)$, $e(A_1) = e(A_1) \land e(A_2) = e(A_1 \land A_2)$, and this isomorphism is not reflected.
% \end{proof}

% \lemma{\label{lemma-scc}By the proof of \ref{thm-conceptual-completeness}, to check that $e$ is an equivalence, it suffices to check the following:
%   \begin{enumerate}
%   \item $e$ is conservative.
%   \item $e$ is full on subobjects.
%   \item $e$ finitely covers $T'$.
%     \end{enumerate}
%   }
  
%   \begin{proof}[Proof of \ref{thm-scc}, item $1$ of \ref{lemma-scc}.]
%     If $e(f) : e(A) \to e(B)$ is an isomorphism, then by definition $e(f)$ is a natural isomorphism between the functors $e(A)$ and $e(B)$. That is, $M(f)$ is a bijection for every model $M$ of $T$.

%     By the completeness theorem for first-order logic, this means that $f$ was an isomorphism in $T$.
%   \end{proof}

%   \begin{proof}[Proof of \ref{thm-scc}, item $2$ of \ref{lemma-scc}.]
%     We begin with a claim:
%     \begin{description}
%     \item[Claim.]
%       \label{lemma-scc-item-2-claim-1} Let $M$ and $N$ be models of $T$. Let $a \in M(A)$ and $b \in N(A)$ for some $A \in T$. Suppose that for all definable subobjects $\Sigma \in \Sub(A)$, whenever $b \in N(\Sigma)$, then we have that $a \in M(\Sigma)$. Then there is an ultrafilter $\mc{U}$ and a homomorphism of structures $h : N \to M^{\mc{U}}$ such that $\Delta_A(a) = h_A(b)$ where $\Delta$ is the diagonal embedding into the ultrapower $M^{\mc{U}}$.
%       \end{description}
%     \end{proof}